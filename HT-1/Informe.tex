\documentclass[11pt]{article}
\usepackage[spanish]{babel}
\usepackage[margin=1in]{geometry}
\usepackage{amsmath, amssymb, amsfonts}
\usepackage{booktabs}
\usepackage{enumitem}
\usepackage{tcolorbox}

\title{\textbf{Hoja de Trabajo 1: Agentes y Entornos} \vspace{1em}\\
Inteligencia Artificial - CC3085}
\author{José Antonio Mérida Castejón}
\date{\today}

\begin{document}
\maketitle

\section*{Task 1: Análisis de Racionalidad y Métricas}

\textit{Imaginen que están diseñando un agente de IA para el sistema de semáforos de la Ciudad de Guatemala
(zona alta de tráfico).}

\begin{enumerate}

  \item \textit{Propongan dos (2) métricas de desempeño distintas para este agente:}

    \begin{itemize}
      \item \textbf{Métrica A (Flujo vehicular):} Podemos empezar con dos opciones bastante \textit{naive}, tales como:
        \begin{itemize}
          \item \textbf{Velocidad promedio de carros cruzando:} Aquí seguimos una linea lógica dónde mientras más tráfico
            hay en la ciudad, menor es la velocidad de los carros. Sin embargo, encontramos algunos problemas cuando nos
            ponemos a pensar sobre la prioridad de las vías principales. Si el agente busca maximizar la velocidad promedio,
            únicamente deben de no frenar (o desacelerar) los carros en las vías principales para aumentar la métrica. Esto
            mientras los carros en vías auxiliares no tienen prioridad alguna, ya que si se formó alguna cola estos carros
            irán despacio y no maximizan la métrica.

          \item \textbf{Cantidad de carros cruzando:} Aquí seguimos una linea de pensamiento bastante lógica, similar al
            inciso anterior. Mientras más carros cruzan intersecciones, significa que tenemos menos tráfico (o un mayor
            flujo vehicular). El problema con esta métrica también es parecido, el modelo no tiene incentiva alguna para
            otorgar paso a una calle auxiliar con $n$ carros estancados si tenemos $2n$ carros transitando por una vía
            principal.
        \end{itemize}

        En resumen, las dos métricas inicialmente propuestas favorecen el volumen bruto e ignoran completamente la equidad
        dentro del sistema. Esto nos lleva a buscar una métrica más justa para las calles con un mejor flujo de tráfico,
        por lo cual debemos buscar \textit{normalizar} estas medidas de cierta manera. Dando como resultado:

        \begin{itemize}
        \item \textbf{Ocupación de carril:} En lugar de contar la cantidad de autos cruzando, buscamos medir que porcentaje
          del tramo regulado por el semáforo se encuentra ocupado. Aquí logramos evitar el problema dónde tener 25 carros
          en una pequeña via auxiliar es completamente diferente a tener 25 carros en un bulevar. Para implementar la métrica
          dentro de nuestro modelo, debemos buscar un punto crítico (por ejemplo, 80\% o 90\%) basándonos en observaciones
          de la vida real para determinar cuando se forma un \textit{gridlock}. Buscando penalizar de manera exponencial
          los estancamientos, podemos proponer la siguiente ecuación:

          \begin{equation}
            C(O) = \frac{1}{(O_{crit} - O)^k}
          \end{equation}

          Dónde:

          \vspace{1em}

            \begin{itemize}
              \item \textbf{$O$}: Representa la \textit{ocupación actual} del carril
              \item \textbf{$O_{crit}$}: Es el \textit{punto crítico de saturación}
              \item \textbf{$k$}: Actúa como un \textit{parámetro de sensibilidad}
              \item \textbf{$C(O)$}: Es la \textit{función de costo} resultante.
            \end{itemize}

          \vspace{1em}
          
            Adicionalmente, debemos tomar en cuenta que los parámetros de esta función costo siempre deben basarse en
            datos de la vida real. Por ejemplo, utilizar un $k$ alto o $O_{crit}$ bajo sería totalmente contraproducente
            en una zona de alto tráfico ya que los deadlocks son inevitables. Sin embargo, la mayor crítica es que esta
            métrica no se podría utilizar por si sola. Podemos darnos cuenta que la manera más eficiente de minimizar
            el costo es simplemente no dejar ingresar ningún vehículo al sistema. Si no tenemos ningún carro dentro de la
            zona, el costo es 0 y el agente ha cumplido exitosamente su misión.

          \item \textbf{Throughput:} Aquí buscamos darle una recompensa al modelo, de manera que no se encuentre en estado
            de parálisis con la finalidad de evitar penalizaciones. Podemos iniciar con la métrica de número de carros 
            cruzando, pero $normalizándola$ respecto al tiempo de luz verde. Definiendo esta métrica por medio de la
            siguiente ecuación:

            \begin{equation}
              T = \frac{N}{t_{g}}
            \end{equation}

          Dónde:

          \vspace{1em}

            \begin{itemize}
              \item \textbf{$T$}: Representa el \textit{throughput}
              \item \textbf{$N$}: Representa la \textit{cantidad de carros que cruzaron en el intervalo de luz verde}
              \item \textbf{$t_{g}$}: Representa la \textit{longitud del intervalo de luz verde}
            \end{itemize}

          \vspace{1em}

          Adicionalmente, podríamos asignar pesos diferentes a áreas con mayor transito para evitar sesgos dentro del sistema.
          Sin embargo, el \textit{fine tuning} de los parámetros queda fuera del alcance de esta hoja de trabajo. También
          debemos tomar en cuenta que buscamos un balance entre ambas métricas descritas, evitando los estancamientos y
          priorizando una cantidad alta de autos fluyendo. Esto nos lleva a la métrica final.

        \item \textbf{Indice de Fluidez:} En esta métrica que nombramos Índice de Fluidez ($IDF$) buscamos 
          tomar en cuenta el \textit{Throughput} y \textit{Ocupación de Carril} y para medir el desempeño de un sistema
          de semáforos inteligentes. Tomando las ecuaciones (1) y (2), podemos construir la siguiente ecuación:

          \begin{equation}
            IDF = {w_{1}}T - {w_{2}}C(O)
          \end{equation}

          Dónde:

          \vspace{1em}

          \begin{itemize}
            \item \textbf{$IDF$}: Representa el \textit{índice de fluidez} del sistema en una intersección específica.
            \item \textbf{${w_{1}}$}: Representa el \textit{peso de Throughput} del sistema.
            \item \textbf{$T$}: Representa el \textit{Throughput} del sistema en una intersección específica.
            \item \textbf{${w_{2}}$}: Representa el \textit{peso de la Ocupación} del sistema.
            \item \textbf{$C(O)$}: Representa la \textit{Ocupación} de las vías en una intersección específica.
          \end{itemize}

          \vspace{1em}

          Esta métrica nos permite medir el desempeño en intersecciones específicas, al igual que a lo largo de toda
          la red en una zona específica. En un mundo ideal, el agente es capaz de calcular esta métrica de rendimiento
          y tomar las decisiones que la minimicen en tiempo real (o en cada ciclo). Sin embargo, también es importante
          tomar en cuenta que cuenta con diferentes parámetros que deben ser escogidos por medio de decisiones informadas.
          Antes de queres implementar un sistema, se debería de obtener información sobre el tráfico en Guatemala (más
          específicamente, en la zona de implementación) y diferentes \textit{edge cases} para asegurarnos que el agente
          tome decisiones beneficiosas para el tráfico.

        \end{itemize}

      \item \textbf{Métrica B (Seguridad peatonal):} [cite: 17]
    \end{itemize}

  \item \textit{Describa un escenario específico dónde en agente actuaría de manera "racional" bajo la métrica
    A, pero esta acción sería considerada un desastre bajo la Métrica B.}

    \begin{itemize}
      \item Texto de pruebna
    \end {itemize}

\end{enumerate}

\section*{Task 2: PEAS y Entornos}

  \textit{Consideren un Robot Autónomo de Limpieza de Páneles Solares instalado
en una granja solar en el desierto de Zacapa. El robot se mueve sobre rieles instalados en los páneles, tiene
una cámara para detectar suciedad/daños, un brazo con cepillo y agua, y una conexión a internet para recibir
reportes del clima. En base a esto, responda lo siguiente:}

\begin{enumerate}

  \item \textit{Complete la siguiente descripción del agente:}

\begin{description}[style=nextline, leftmargin=0pt, font=\bfseries]
  \item[P (Performance/Desempeño): \textnormal{\textit{(Mencione al menos 3 indicadores cuantificables)}}] 
    
  \item[E (Environment/Entorno): \textnormal{\textit{(¿Qué rodea al agente y con qué interactúa?)}}] 

  \item[A (Actuators/Actuadores): \textnormal{\textit{(Mecanismos para afectar el entorno)}}] 

  \item[S (Sensors/Sensores): \textnormal{\textit{(Entradas de percepción). Asumiendo que no hace ningún
        preprocesamiento, ¿qué métrica usaría para medir el desempeño de su modelo?}}] 
\end{description}

  \item \textit{Clasifique el entorno de este robot según las 4 dimensiones vistas en clase. Debe
    justificar cada elección (una elección sin justificación tiene valor de 0pts).}
        
    \begin{enumerate}[(a)]
      \item \textit{¿Completamente o Parcialmente observable?}
        \textbf{Respuesta.} Justificacion
      

      \item \textit{¿Determinístico o Estocástico?}

      \item \textit{¿Discreto o Continuo?}

      \item \textit{¿Benigno o Adverso?}

\end{enumerate}

\section*{Task 3: Modelado}

Representación del estado: $f(\text{estado}) \rightarrow \text{acción}$[cite: 47].

\section*{Task 4: Práctica - Agente Reflejo Simple}

El código para las clases \texttt{Environment} y \texttt{Agent} se adjunta por separado[cite: 54, 60, 79].

\end{document}
