\documentclass[11pt]{article}
\usepackage[spanish]{babel}
\usepackage[margin=1in]{geometry}
\usepackage{amsmath, amssymb, amsfonts}
\usepackage{booktabs}
\usepackage{enumitem}
\usepackage{tcolorbox}

\title{\textbf{Hoja de Trabajo 1: Agentes y Entornos} \vspace{1em}\\
Inteligencia Artificial - CC3085}
\author{José Antonio Mérida Castejón}
\date{\today}

\begin{document}
\maketitle

\section*{Task 1: Análisis de Racionalidad y Métricas}

\textit{Imaginen que están diseñando un agente de IA para el sistema de semáforos de la Ciudad de Guatemala
(zona alta de tráfico).}

\begin{enumerate}

  \item \textit{Propongan dos (2) métricas de desempeño distintas para este agente:}

    \begin{itemize}
      \item \textbf{Métrica A (Flujo vehicular):} Podemos empezar con dos opciones bastante \textit{naive}, tales como:
        \begin{itemize}
          \item \textbf{Velocidad promedio de carros cruzando:} Aquí seguimos una linea lógica dónde mientras más tráfico
            hay en la ciudad, menor es la velocidad de los carros. Sin embargo, encontramos algunos problemas cuando nos
            ponemos a pensar sobre la prioridad de las vías principales. Si el agente busca maximizar la velocidad promedio,
            únicamente deben de no frenar (o desacelerar) los carros en las vías principales para aumentar la métrica. Esto
            mientras los carros en vías auxiliares no tienen prioridad alguna, ya que si se formó alguna cola estos carros
            irán despacio y no maximizan la métrica.

          \item \textbf{Cantidad de carros cruzando:} Aquí seguimos una linea de pensamiento bastante lógica, similar al
            inciso anterior. Mientras más carros cruzan intersecciones, significa que tenemos menos tráfico (o un mayor
            flujo vehicular). El problema con esta métrica también es parecido, el modelo no tiene incentiva alguna para
            otorgar paso a una calle auxiliar con $n$ carros estancados si tenemos $2n$ carros transitando por una vía
            principal.
        \end{itemize}

        En resumen, las dos métricas inicialmente propuestas favorecen el volumen bruto e ignoran completamente la equidad
        dentro del sistema. Esto nos lleva a buscar una métrica más justa para las calles con un mejor flujo de tráfico,
        por lo cual debemos buscar \textit{normalizar} estas medidas de cierta manera. Dando como resultado:

        \begin{itemize}
        \item \textbf{Ocupación de carril:} En lugar de contar la cantidad de autos cruzando, buscamos medir que porcentaje
          del tramo regulado por el semáforo se encuentra ocupado. Aquí logramos evitar el problema dónde tener 25 carros
          en una pequeña via auxiliar es completamente diferente a tener 25 carros en un bulevar. Para implementar la métrica
          dentro de nuestro modelo, debemos buscar un punto crítico (por ejemplo, 80\% o 90\%) basándonos en observaciones
          de la vida real para determinar cuando se forma un \textit{gridlock}. Buscando penalizar de manera exponencial
          los estancamientos, podemos proponer la siguiente ecuación:

          \begin{equation}
            C(O) = \frac{1}{(O_{crit} - O)^k}
          \end{equation}

          Dónde:

          \vspace{1em}

            \begin{itemize}
              \item \textbf{$O$}: Representa la \textit{ocupación actual} del carril
              \item \textbf{$O_{crit}$}: Es el \textit{punto crítico de saturación}
              \item \textbf{$k$}: Actúa como un \textit{parámetro de sensibilidad}
              \item \textbf{$C(O)$}: Es la \textit{función de costo} resultante.
            \end{itemize}

          \vspace{1em}
          
            Adicionalmente, debemos tomar en cuenta que los parámetros de esta función costo siempre deben basarse en
            datos de la vida real. Por ejemplo, utilizar un $k$ alto o $O_{crit}$ bajo sería totalmente contraproducente
            en una zona de alto tráfico ya que los deadlocks son inevitables. Sin embargo, la mayor crítica es que esta
            métrica no se podría utilizar por si sola. Podemos darnos cuenta que la manera más eficiente de minimizar
            el costo es simplemente no dejar ingresar ningún vehículo al sistema. Si no tenemos ningún carro dentro de la
            zona, el costo es 0 y el agente ha cumplido exitosamente su misión.

          \item \textbf{Throughput:} Aquí buscamos darle una recompensa al modelo, de manera que no se encuentre en estado
            de parálisis con la finalidad de evitar penalizaciones. Podemos iniciar con la métrica de número de carros 
            cruzando, pero $normalizándola$ respecto al tiempo de luz verde. Definiendo esta métrica por medio de la
            siguiente ecuación:

            \begin{equation}
              T = \frac{N}{t_{g}}
            \end{equation}

          Dónde:

          \vspace{1em}

            \begin{itemize}
              \item \textbf{$T$}: Representa el \textit{throughput}
              \item \textbf{$N$}: Representa la \textit{cantidad de carros que cruzaron en el intervalo de luz verde}
              \item \textbf{$t_{g}$}: Representa la \textit{longitud del intervalo de luz verde}
            \end{itemize}

          \vspace{1em}

          Adicionalmente, podríamos asignar pesos diferentes a áreas con mayor transito para evitar sesgos dentro del sistema.
          Sin embargo, el \textit{fine tuning} de los parámetros queda fuera del alcance de esta hoja de trabajo. También
          debemos tomar en cuenta que buscamos un balance entre ambas métricas descritas, evitando los estancamientos y
          priorizando una cantidad alta de autos fluyendo. Esto nos lleva a la métrica final.

        \item \textbf{Indice de Fluidez:} En esta métrica que nombramos Índice de Fluidez ($IDF$) buscamos 
          tomar en cuenta el \textit{Throughput} y \textit{Ocupación de Carril} y para medir el desempeño de un sistema
          de semáforos inteligentes. Tomando las ecuaciones (1) y (2), podemos construir la siguiente ecuación:

          \begin{equation}
            IDF = {w_{1}}T - {w_{2}}C(O)
          \end{equation}

          Dónde:

          \vspace{1em}

          \begin{itemize}
            \item \textbf{$IDF$}: Representa el \textit{índice de fluidez} del sistema en una intersección específica.
            \item \textbf{${w_{1}}$}: Representa el \textit{peso de Throughput} del sistema.
            \item \textbf{$T$}: Representa el \textit{Throughput} del sistema en una intersección específica.
            \item \textbf{${w_{2}}$}: Representa el \textit{peso de la Ocupación} del sistema.
            \item \textbf{$C(O)$}: Representa la \textit{Ocupación} de las vías en una intersección específica.
          \end{itemize}

          \vspace{1em}

          Esta métrica nos permite medir el desempeño en intersecciones específicas, al igual que a lo largo de toda
          la red en una zona específica. En un mundo ideal, el agente es capaz de calcular esta métrica de rendimiento
          y tomar las decisiones que la minimicen en tiempo real (o en cada ciclo). Sin embargo, también es importante
          tomar en cuenta que cuenta con diferentes parámetros que deben ser escogidos por medio de decisiones informadas.
          Antes de queres implementar un sistema, se debería de obtener información sobre el tráfico en Guatemala (más
          específicamente, en la zona de implementación) y diferentes \textit{edge cases} para asegurarnos que el agente
          tome decisiones beneficiosas para el tráfico.

        \end{itemize}

      \item \textbf{Métrica B (Seguridad peatonal):} Similar al inciso anterior, primero vamos a indagar sobre algunas
        métricas \textit{naive} e ir iterando para obtener una métrica más adecuada.

        \begin{itemize}
          \item \textbf{Conteo de Colisiones:} Esta primera idea parece ser un buen acercamiento, sin embargo podemos darnos
            cuenta rápidamente que no es apta para un sistema como este. Realmente, esta es una métrica reactiva y un agente
            necesita que ocurra un accidente para recibir una penalización. Adicionalmente, estos eventos deberían ser pocos
            y las condiciones que los dan pueden no ser consistentes como para que el agente realmente \textit{aprenda}.

          \item \textbf{Tiempo de Respuesta Botón Peatonal:} Dentro del sistema de tráfico, podemos asumir que una persona
            es más propensa a realizar alguna acción peligrosa (por ejemplo, cruzar la calle sin tener el paso) mientras
            más tiempo pasa esperando. Esto implica que un botón de paso peatonal con un tiempo de respuesta menor resulta
            en una mayor seguridad peatonal. El problema más evidente aquí, es que es completamente incompatible el manejo
            de fluidez vehicular con un botón peatonal \textit{óptimo} que otorgue paso cuando sea posible. Adicionalmente,
            el cambio constante de estados de luz verde / roja puede generar inseguridad para tanto peatones cómo vehículos.

        \end{itemize}

        Adicionalmente, también debemos de tomar en cuenta que un sistema de inteligencia artificial por defecto no toma
        decisiones similares a las de un ser humano. Es decir, todo lo que \textit{sabe} se le debe enseñar. Esto incluye
        razonamientos triviales, cómo lo puede ser \textit{``No coloquemos el semáforo en verde si acabamos de colocarlo
        en rojo para el cruce de este peatón``}. Esto nos lleva a definir las siguientes dos métricas auxiliares:

        \begin{itemize}
          \item \textbf{Factor de Suficiencia:} Esta métrica mide si el tiempo de verde es físicamente suficiente para que
            un grupo de peatones termine de cruzar. Aquí la métrica es bastante simple, dónde podemos definir la métrica
            por medio de la siguiente ecuación:

          \begin{equation}
            S = {t_{verde}} - {t_{despeje}}
          \end{equation}

          Dónde

          \vspace{1em}

          \begin{itemize}
            \item \textbf{$S$}: Es el \textit{factor de suficiencia}
            \item \textbf{$t_{verde}$}: Es el \textit{tiempo que la luz peatonal estuvo en verde}
            \item \textbf{$t_{despeje}$}: Es el \textit{tiempo para que se despejara la calle}
          \end{itemize}

          \vspace{1em}

          Aquí se podría argumentar que lo mejor sería un sistema dónde no se permitiera el cambio mientras un peatón este
          cruzando, sin embargo una persona a media calle podría provocar un estancamiento total del tráfico al siempre
          tener el paso.


          \item \textbf{Factor de Espera Crítica:} De manera análoga a la métrica de \textit{Ocupación de carril}, aquí vamos
          a modelar el \textit{Factor de Desesperación} como una función que crece exponencialmente conforme el tiempo de espera
          se acerca a un punto crítico de paciencia del peatón. Definimos la ecuación de la siguiente manera:

          \begin{equation}
            E(t_w) = \frac{1}{(t_{crit} - t_w)^k}
          \end{equation}

          Dónde:

          \vspace{1em}

          \begin{itemize}
            \item \textbf{$t_w$}: Representa el \textit{tiempo de espera actual} del peatón (desde el accionamiento del botón).
            \item \textbf{$t_{crit}$}: Es el \textit{tiempo crítico de paciencia}, donde el peatón decide cruzar ilegalmente.
            \item \textbf{$k$}: Actúa como un \textit{parámetro de sensibilidad} al tiempo.
            \item \textbf{$E(t_w)$}: Es la \textit{función de costo por desesperación} resultante.
          \end{itemize}

          \vspace{1em}

          Aquí debemos de ser cuidadosos al elegir los parámetros, ya que deberían de reflejar la vida real de la manera
          más cercana posible. De hecho, antes de implementar esta función la primera recomendación sería verificar los
          supuestos sobre los cuales trabaja el modelo con información del mundo real. Por último, tenemos una métrica
          que busca combinar los factores anteriores.

        \item \textbf{Índica de Seguridad Peatonal:} El \textit{Índice de Seguridad Peatonal} o $ISP$, busca minimizar los
          tiempos de espera peatonales al igual que asegurar tiempos correctos de despeje de vía. Esta métrica penaliza
          al modelo si el tiempo de verde otorgado es suficiente para que los peatones completen el cruce, y busca penalizar
          tiempos de espera cercanos al punto crítico que incentiva al peatón a cruzar sin paso. Tomando las ecuaciones (4)
          y (5), lo modelamos por medio de la siguiente ecuación:

        \begin{equation}
            ISP = w_{3}S - w_{4}E(t_w)
        \end{equation}

          Dónde:

        \vspace{1em}

        \begin{itemize}
          \item \textbf{$ISP$}: Representa el \textit{índice de seguridad peatonal} en el cruce.
          \item \textbf{$w_3$}: Es el peso asignado al \textit{Factor de Suficiencia}.
          \item \textbf{$S$}: Representa la \textit{Suficiencia de Despeje} ($t_{verde} - t_{despeje}$).
          \item \textbf{$w_4$}: Es el peso asignado al \textit{Factor de Espera Crítica}.
          \item \textbf{$E(t_w)$}: Es la función de costo por \textit{Desesperación} definida anteriormente.
        \end{itemize}

      \vspace{1em}


    \end{itemize}
  \end{itemize}
  Como reflexión o conclusión de este inciso, implementar un sistema inteligente de control de tráfico en Guatemala sería
  un proyecto de complejidad alta. Algo básico como definir métricas de desempeño puede volverse complicado, teniendo que tomar
  en cuenta diferentes factores como:

    \begin{itemize}
      \item Situaciones de tráfico específicas a la zona.
      \item Comportamiento de las métricas en \textit{edge cases}.
      \item Viabilidad de implementación de sensores.
      \item Balance en priorización de diferentes factores, cómo seguridad peatonal y flujo vehicular.
    \end{itemize} 

  Adicionalmente, los supuestos detallados en cada métrica propuesta definitivamente deben ser validados.

  \item \textit{Describa un escenario específico dónde en agente actuaría de manera ``racional`` bajo la métrica
    A, pero esta acción sería considerada un desastre bajo la Métrica B.}

    A pesar de haber iterado y analizado extensivamente las métricas, considero que el sistema debe de ser probado
    exhaustivamente para poder llegar a un \textit{punto de equilibro} dónde no le de prioridad a una única métrica
    sobre la otra. Ya que en la mayoría de los casos, un desbalance entre peatones y vehículos puede resultar en
    desastres absolutos.  

    Por ejemplo: Bajo la métrica $A$, se llega a un punto crítico $O_{crit}$ de atascamiento que aumenta exponencialmente.
    Siendo un día lluvioso, fin de semana y quincena en una intersección con una vía principal sumamente
    transcurrida cercana a $O_{crit}$. La acción más racional del agente, es intentar minimizar la penalización dada por
    $C(O)$ y dar paso a la mayor cantidad de carros posible por esta vía principal y maximizar si Índice de Fluidez ($IDF$).
    Esto también aumentaría el factor $Throughput$ ($T$), creando una situación dónde el comportamiento obvio es quitarle
    prioridad a los peatones. Mientras tanto, los peatones se encuentran esperando con un \textit{Factor de Espera} ($E(t_{w}$))
    crítico, aumentando exponencialmente mientras el tiempo sigue transcurriendo. Adicionalmente, digamos que el agente decide
    por fin que los tiempos de espera peatonales han sido suficientes. Se ha dado paso peatonal, pero mientras los peatones
    están cruzando la calle, la vía principal se atascó nuevamente y se ha llegado a $O_{crit}$. Aquí se puede llegar a
    cortar la vía peatonal, generando aún otra penalización, esta vez del \textit{Factor de Suficiencia} ($S$). En este caso,
    se tomó la acción óptima en cuanto a la métrica $A$, pero la métrica $B$ se vio penalizada fuertemente tomando en cuenta
    ambos factores.


  \item \textit{Basándose en la diapositiva de Incertidumbre, explique por qué este entorno de tráfico nunca podría
    ser Completamente Observable y cómo la ``Limitante en sensores`` afecta la racionalidad de su agente}


    El entorno del tráfico en Guatemala siempre será \textit{Parcialmente Observable}, ya que un agente jamás tendrá acceso
    total a un estado global del sistema en tiempo real. En el caso de este modelo, buscamos tener acceso a la mayor cantidad
    de información posible pero siempre nos veremos limitados. Por ejemplo, podemos identificar los siguientes sensores y
    sus limitantes:

    \begin{itemize}
      \item Modelos de Visión por Computadora: Si un peatón o moto es tapado por un vehículo pesado, somos incapaces
        de percibir el estado del sistema.

      \item Sensores de Ocupación: Los sensores están físicamente limitados a su posición, la ocupación de carril por 
        ejemplo, jamás podrá ser medida en un intervalo contínuo de manera perfecta.
    \end{itemize}

    Tomando también en cuenta, que tenemos algunos otros tipos de sensores que simplemente no serían viables en esta situación.
    Por ejemplo, si tomáramos modelos más pesados o complejos como de mapeo 3D del entorno, el procesamiento de estos datos
    tendría un costo computacional demasiado alto para el modelo.

    Esta limitante en los sensores básicamente redefine la racionalidad del agente. Este no es juzgado en función del éxito
    absoluto (o verdadero) en el mundo real, si no en función de las percepciones disponibles. Gracias a estas limitaciones, el 
    agente opera bajo una racionalidad limitada. Dónde, por ejemplo, si sus sensores no logran detectar a un peatón cruzando la
    calle, este peatón realmente jamás existió y no pudo existir alguna decisión o penalización asociada. El modelo sigue
    tomando decisiones óptimas y racionales, pero al hacerlo sobre datos incompletos no paran reflejando la decisión óptima
    real. 

\end{enumerate}

\section*{Task 2: PEAS y Entornos}

  \textit{Consideren un Robot Autónomo de Limpieza de Páneles Solares instalado
en una granja solar en el desierto de Zacapa. El robot se mueve sobre rieles instalados en los páneles, tiene
una cámara para detectar suciedad/daños, un brazo con cepillo y agua, y una conexión a internet para recibir
reportes del clima. En base a esto, responda lo siguiente:}

\begin{enumerate}

  \item \textit{Complete la siguiente descripción del agente:}

\begin{description}[style=nextline, leftmargin=0pt, font=\bfseries]
  \item[P (Performance/Desempeño): \textnormal{\textit{(Mencione al menos 3 indicadores cuantificables)}}] ~

    \begin{enumerate}

      \item \textbf{Calidad / Eficiencia de Limpieza:} Esta métrica busca medir que tan bien realiza la limpieza de
        cada panel individual el agente. Aquí prácticamente contamos con dos opciones para realizar las mediciones. Primero,
      podemos utilizar la cámara para identificar el \% de área con suciedad y ver cuanto se logró remover. O segundo, podemos
      comparar la cantidad de energía generada por el panel antes y después de la limpieza. Personalmente, me gusta más la
      segunda opción ya que movemos la capa de abstracción un poco más arriba, dónde permitimos que el agente tome decisiones
      sobre las capas que nos interesan un poco menos (\textit{optimicidad de limpieza}) y se enfoque en la generación energética
      real. Cómo formula de medición, podemos tomar la diferencia en producción de energía:

      \[ \Delta P = P_{post} - P_{pre}\]

      \item \textbf{Tiempo de Limpieza:} Esta métrica busca medir la eficiencia en cuánto a tiempo de
        limpieza por panel individual. Se puede medir de dos maneras, primero el tiempo promedio de limpieza en cada panel
        dentro de un periodo de tiempo. O segundo, la cantidad de páneles limpiados dentro de cierto periodo de tiempo. La
        diferencia yace en que en uno se está tomando en cuenta el periodo de desplazamiento, mientras que en otro no. Por lo cual,
        decidiría ir por la primera en caso que tuviera que implementar el sistema:

        \[ \bar{t} = \frac{1}{n} \sum_{i=1}^{n} (t_{final, i} - t_{start, i}) \]

      \item \textbf{Consumo de Agua:} Esta métrica busca optimizar el uso de agua parte del robot. Esta métrica
        es especialmente clave, considerando que nos encontramos en el desierto de Zacapa y el agua puede ser un recurso
        relativamente escaso. Buscamos optimizar el recurso utilizado por panel para no penalizar al robot por limpiar más
        páneles.

        \[ \bar{v} = \frac{V_{total}}{n} \]


      \item \textbf{Numero de Paneles:} Esta métrica busca aumentar (o por lo menos, generar una incentiva) la cantidad
        de paneles limpiados por el robot en un determinado periodo de tiempo. Si nos regimos únicamente por las métricas anteriores,
        el robot ideal realiza una limpieza súmamente eficiente y deja de operar. Ya que se continua realizando operaciones, puede que
        las métricas bajen.


    \end{enumerate}
    
  \item[E (Environment/Entorno): \textnormal{\textit{(¿Qué rodea al agente y con qué interactúa?)}}] 

  \item[A (Actuators/Actuadores): \textnormal{\textit{(Mecanismos para afectar el entorno)}}] 

  \item[S (Sensors/Sensores): \textnormal{\textit{(Entradas de percepción). Asumiendo que no hace ningún
        preprocesamiento, ¿qué métrica usaría para medir el desempeño de su modelo?}}] 
\end{description}

  \item \textit{Clasifique el entorno de este robot según las 4 dimensiones vistas en clase. Debe
    justificar cada elección (una elección sin justificación tiene valor de 0pts).}
        
    \begin{enumerate}[(a)]
      \item \textit{¿Completamente o Parcialmente observable?}
        \textbf{Respuesta.} Justificacion
      

      \item \textit{¿Determinístico o Estocástico?}

      \item \textit{¿Discreto o Continuo?}

      \item \textit{¿Benigno o Adverso?}

\end{enumerate}

\section*{Task 3: Modelado}

Representación del estado: $f(\text{estado}) \rightarrow \text{acción}$[cite: 47].

\section*{Task 4: Práctica - Agente Reflejo Simple}

El código para las clases \texttt{Environment} y \texttt{Agent} se adjunta por separado[cite: 54, 60, 79].

\end{document}
