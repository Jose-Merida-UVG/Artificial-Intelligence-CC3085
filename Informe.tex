\documentclass[11pt]{article}
\usepackage[spanish]{babel}
\usepackage[margin=1.2in]{geometry}
\usepackage{amsmath, amssymb, amsfonts}
\usepackage{booktabs}
\usepackage{enumitem}
\usepackage{tcolorbox}

\title{\textbf{Hoja de Trabajo 1: Agentes y Entornos}}
\author{José Antonio Mérida Castejón}
\date{\today}

\begin{document}
\maketitle

\section*{Task 1: Análisis de Racionalidad y Métricas}

\textit{Imaginen que están diseñando un agente de IA para el sistema de semáforos de la Ciudad de Guatemala
(zona alta de tráfico).}

\begin{enumerate}

  \item \textit{Propongan dos (2) métricas de desempeño distintas para este agente:}

    \begin{itemize}
      \item \textbf{Métrica A (Flujo vehicular):} [cite: 16]
      \item \textbf{Métrica B (Seguridad peatonal):} [cite: 17]
    \end{itemize}

  \item \textit{Describa un escenario específico dónde en agente actuaría de manera "racional" bajo la métrica
    A, pero esta acción sería considerada un desastre bajo la Métrica B.}

    \begin{itemize}
      \item Texto de pruebna
    \end {itemize}

\end{enumerate}

\section*{Task 2: PEAS y Entornos}

  \textit{Consideren un Robot Autónomo de Limpieza de Páneles Solares instalado
en una granja solar en el desierto de Zacapa. El robot se mueve sobre rieles instalados en los páneles, tiene
una cámara para detectar suciedad/daños, un brazo con cepillo y agua, y una conexión a internet para recibir
reportes del clima. En base a esto, responda lo siguiente:}

\begin{enumerate}

  \item \textit{Complete la siguiente descripción del agente:}

\begin{description}[style=nextline, leftmargin=0pt, font=\bfseries]
  \item[P (Performance/Desempeño): \textnormal{\textit{(Mencione al menos 3 indicadores cuantificables)}}] 
    
  \item[E (Environment/Entorno): \textnormal{\textit{(¿Qué rodea al agente y con qué interactúa?)}}] 

  \item[A (Actuators/Actuadores): \textnormal{\textit{(Mecanismos para afectar el entorno)}}] 

  \item[S (Sensors/Sensores): \textnormal{\textit{(Entradas de percepción). Asumiendo que no hace ningún
        preprocesamiento, ¿qué métrica usaría para medir el desempeño de su modelo?}}] 
\end{description}

  \item \textit{Clasifique el entorno de este robot según las 4 dimensiones vistas en clase. Debe
    justificar cada elección (una elección sin justificación tiene valor de 0pts).}
        
    \begin{enumerate}[(a)]
      \item \textit{¿Completamente o Parcialmente observable?}
        \textbf{Respuesta.} Justificacion
      

      \item \textit{¿Determinístico o Estocástico?}

      \item \textit{¿Discreto o Continuo?}

      \item \textit{¿Benigno o Adverso?}

\end{enumerate}

\section*{Task 3: Modelado}

Representación del estado: $f(\text{estado}) \rightarrow \text{acción}$[cite: 47].

\section*{Task 4: Práctica - Agente Reflejo Simple}

El código para las clases \texttt{Environment} y \texttt{Agent} se adjunta por separado[cite: 54, 60, 79].

\end{document}
