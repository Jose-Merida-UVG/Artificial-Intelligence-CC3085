\documentclass[11pt]{article}
\usepackage[spanish]{babel}
\usepackage[margin=1in]{geometry}
\usepackage{amsmath, amssymb, amsfonts}
\usepackage{booktabs}
\usepackage{enumitem}
\usepackage{tcolorbox}
\usepackage{hyperref}

\title{\textbf{Laboratorio 1: Métricas de Desempeño} \vspace{1em}\\
Inteligencia Artificial - CC3085}
\author{José Antonio Mérida Castejón}
\date{\today}

\begin{document}
\maketitle

\section*{Task 1: El Dilema del Negocio}
\textit{Imagine que usted ha sido contratado como Lead Data Scientist para tres startups diferentes. Para cada caso,
responda las preguntas justificando su respuesta.}
  \subsection*{Caso A: ``MediScan AI``}
  \textit{Están desarrollando un modelo para detectar un tipo de cáncer raro en etapas tempranas a partir de radiografías.}
    \begin{enumerate}
      \item \textit{En este contexto, ¿qué es peor: un \textbf{Falso Positivo} o \textbf{Falso Negativo}?}

        \vspace{0.5em}\\
        En este caso, un \textbf{falso negativo} es significativamente más peligroso (o \textit{peor}). En el caso de un
        \textbf{falso negativo}, el paciente abandona completamente la posibilidad de tener la enfermedad y se elimina cualquier
        posibilidad de recibir tratamiento oportuno. Mientras que en un \textbf{falso positivo}, el paciente continúa con el monitoreo
        y exámenes necesarios a pesar de no tener la enfermdad. Una opción genera gastos monetarios en exámenes (o tratamientos),
        mientras que el otro resulta siendo potencialmente mortal.

        En mi opinión, el modelo debería de tener un \textit{threshold} sumamente bajo para indicar una potencial presencia
        de la enfermedad. La información proveída por el modelo podría servir como un tipo de \textit{screening} temprano, seguido
        de un régimen establecido de exámenes para continuar con el monitoreo del paciente. En caso que la tecnología se vea limitada
        las unicas opciones fueran \textit{tratamiento} o \textit{no tratamiento}, seguiría considerando peor un \textbf{falso negativo}
        aunque se deberían de examinar los efectos secundarios o demás consecuencias del tratamiento.

      \item \textit{Basado en lo anterior, si tuviera que optimizar el modelo priorizando una sola métrica entre
        \textbf{Precisión (Precision)} y \textbf{Sensibilidad (Recall / Sensitivity)}, ¿cuál escogería y por qué?}

        \vspace{0.5em}\\
        Definitivamente sería \textbf{recall}. Utilizando precisión, el modelo se estaría enfocando en minimizar
        los \textbf{falsos positivos}. Por otro lado, el \textbf{recall} busca minimizar los \textbf{falsos negativos}.
        En otras palabras, utilizando \textbf{recall} estamos entrenando al modelo para identificar la mayor cantidad
        de casos de cáncer correctamente sin importar que tan seguro esté, mientras que utilizando \textbf{precision} lo
        estamos entrenando a priorizar la certeza absoluta antes de indicar cáncer. Esto sería como un doctor que observa
        ciertas indicaciones de la enfermedad, dónde priorizar \textbf{precision} incentiva al doctor a ignorar los
        indicadores y no informar al paciente (negligencia, si me preguntan mi opinón).

      \item \textit{¿Por qué el Accuracy sería una métrica peligrosa para presentar a los inversionistas en este
        caso específico?}
        
        \vspace{0.5em}\\
        Primero que nada, el \textbf{accuracy} realmente no cuenta la historia completa. Para cualquier modelo que estemos
        diseñando, es clave tomar múltiples métricas en cuenta a la hora de analizar su rendimiento y utilidad en el mundo
        real. Esto cubre las métricas de \textbf{recall} y \textbf{precision} discutidas anteriormente, dónde buscamos
        analizar más allá y descubrir un poco más sobre que tipos de errores comete el modelo. Luego, podemos considerar que
        el \textbf{accuracy} de por si es sumamente inapropiado debido a la rareza de la enfermedad que buscamos predecir. Si
        tuviésemos 1 radiografía (o paciente) de cada 20 presentando esta enfermedad, podríamos obtener un \textbf{accuracy}
        del 95\% simplemente prediciendo que ninguno de los pacientes está enfermo. Por último, como discutimos anteriormente,
        tenemos una situación de la vida real dónde las consecuencias de cada tipo de error son completamente diferentes. El
        presentar esta métrica a los accionistas sería un error total, ya que no representa adecuadamente el rendimiento del
        modelo \textit{respecto al problema que se busca resolver}.

    \end{enumerate}

    \subsection*{Caso B: ``SpamGuard`` (Filtro de Correos)}
    \textit{Están creando un filtro de spam para una corporación grande.}

      \begin{enumerate}
        \item \textit{¿Qué error causaría más molestia y pérdida de productividad a los empleados: que un correo de spam llegue
          al inbox (FP o FN dependiendo de su definición) o que un correo importante de un cliente se vaya a la carpeta de spam?}

          En este caso, un correo de spam llegando al inbox sería una inconveniencia ligera para el trabajador que lo reciba. Mientras
          tanto, si un correo importante va a la carpeta de spam tenemos una situación catastrófica. El no ver el correo puede tener
          consecuencias dentro de la empresa, económicas o hasta legales para la persona que lo reciba. Adicionalmente, esta incertidumbre
          llevaría a que el resto de empleados estuviesen revisando la carpeta de spam constantemente. Como resultante, existiría
          una pérdida de productividad significativa, dónde el modelo contribuye al problema en vez de resolverlo.

        \item \textit{¿Qué métrica priorizaría aquí: \textbf{Precision} o \textbf{Recall}? (Defina cual es su clase positiva)}

          Empezamos definiendo \textit{spam} como nuestra clase positiva. En este caso, definitivamente se debería priorizar
          el \textbf{precision} en lugar de \textbf{recall}. Utilizando una analogía, digamos que tenemos una persona encargada
          de revisar cada correo individualmente. Priorizando \textbf{precision}, esta persona clasificaría un correo como spam
          si tuviera certeza que un correo pertenece a esta categoría. Si fuéramos en cambio por \textbf{recall}, esta persona
          elegiría clasificar como spam a cualquier correo que tuviera índices de serlo.
      \end{enumerate}

    \subsection*{Caso C: ``Zillow 2.0`` (Predicción de Precios de Casas)}
    \textit{Están prediciendo el valor de mercado de propiedades. Tienen un modelo con las variables: Metros Cuadrados, Ubicación,
    Número de Cuartos. Ahora, un junior engineer sugiere agregar 50 variables nuevas (como ``color de la puerta``, ``nombre del
    dueño anterior``, etc.)} y nota que el \textbf{$R^2$} subió ligeramente.

      \begin{enumerate}
        \item \textit{¿Deberíamos confiar en este aumento del $R^2$ para decir que el modelo es mejor?}

          \textbf{No.} Primero, al diseñar modelos debemos de tomar en cuenta la \textit{explicabilidad} y no únicamente el rendimiento.
          Puede ser favorable tener un modelo con rendimiento similar pero con menos variables, así podemos identificar más fácilmente
          \textit{por que} y \textit{como} funciona el modelo. Luego, la métrica de de $R^2$  sigue aumentando al agregar nuevas
          variables independientes, sin importar cuán relacionadas estén con nuestra variable objetivo. De manera que, aunque prefiriésemos
          un modelo con una cantidad de variables considerablemente mayor por un incremento pequeño de rendimiento, este incremento
          es un derivado de las variables adicionales y no refleja adecuadamente la calidad del modelo. Por último, deberíamos
          de tener bastante claro que algunas de estas variables muy probablemente solo estén agregando ruido al modelo. No necesitamos
          ser expertos en el tema para darnos cuenta que probablemente no exista una correlación significativa entre el nombre del
          dueño anterior y el valor de mercado de una propiedad. Esto también resalta la necesidad de un EDA, dónde las decisiones a
          tomar al diseñar el modelo deben de estar bien fundamentadas.

        \item \textit{¿Qué métrica debería observar para saber si esas nuevas variables aportan valor o son solo ruido? Justifíquese
          basándose en la métrica de $R^2$ ajustado.}

            La métrica a observar sería $R^2 \ ajustado$. Aquí se modifica la fórmula, agregando una penalización por agregar
            variables nuevas. La fórmula modificada es la siguiente:

              \[Adjusted \ R^2 = 1 - \frac{(1-R^2)(N-1)}_{N - p - 1}\]

            Donde:

            \begin{itemize}
              \item \textbf{$R^2$}: es el $R^2$ de la muestra
              \item \textbf{$N$}: es el tamaño de la muestra
              \item \textbf{$p$}: es el número de variables independientes
            \end{itemize}

            Esto significa que, al agregar una variable el denominador {$N-p-1$} se vuelve más pequeño. Al dividir dentro de un valor
            más pequeño, estamos volviendo la cifra del lado derecho más grande y esto resulta en un $R^2 \ ajustado$ más pequeño. Esto
            resulta en que las variables adicionales puedan \textit{disminuir} el rendimiento del modelo si no proveen valor, mientras
            que utilizando $R^2$ únicamente pueden mantenerlo igual o aumentarlo. Esto quiere decir que para tener un $R^2 \ ajustado$
            más alto, cada variable independiente debe proveer un valor más alto que la penalización dada por el ajuste.

            Adicionalmente, para el entrenamiento del modelo podemos tomar en cuenta diferentes técnicas de regularización como
            \textit{Lasso (L1)} o \textit{Ridge (L2)}. En este caso, personalmente optaría por \textit{Lasso} para ayudar a eliminar
            algunas de las variables nuevas del junior engineer (al ser la regularización más \textit{agresiva}) y darle algo de
            validez a su sugerencia de implementar nuevas variables. De esta manera podemos explorar brevemente algunas sugerencias,
            manteniendo clara la visión de optimizar en cuanto a la métrica $R^2 \ ajustado$.
      \end{enumerate}
\end{document}
