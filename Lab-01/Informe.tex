\documentclass[11pt]{article}
\usepackage[spanish]{babel}
\usepackage[margin=1in]{geometry}
\usepackage{amsmath, amssymb, amsfonts}
\usepackage{booktabs}
\usepackage{enumitem}
\usepackage{tcolorbox}
\usepackage{hyperref}

\title{\textbf{Laboratorio 1: Métricas de Desempeño} \vspace{1em}\\
Inteligencia Artificial - CC3085}
\author{José Antonio Mérida Castejón}
\date{\today}

\begin{document}
\maketitle

\section*{Task 1: El Dilema del Negocio}
\textit{Imagine que usted ha sido contratado como Lead Data Scientist para tres startups diferentes. Para cada caso,
responda las preguntas justificando su respuesta.}
  \subsection*{Caso A: ``MediScan AI``}
  \textit{Están desarrollando un modelo para detectar un tipo de cáncer raro en etapas tempranas a partir de radiografías.}
    \begin{enumerate}
      \item \textit{En este contexto, ¿qué es peor: un \textbf{Falso Positivo} o \textbf{Falso Negativo}?}

        \vspace{0.5em}\\
        En este caso, un \textbf{Falso Negativo} es significativamente más peligroso (o \textit{peor}). En el caso de un
        \textbf{Falso Negativo}, el paciente abandona completamente la posibilidad de tener la enfermedad y se elimina cualquier
        posibilidad de recibir tratamiento oportuno. Mientras que en un \textbf{Falso Positivo}, el paciente continúa con el monitoreo
        y exámenes necesarios a pesar de no tener la enfermdad. Una opción genera gastos monetarios en exámenes (o tratamientos),
        mientras que el otro resulta siendo potencialmente mortal.

        En mi opinión, el modelo debería de tener un \textit{threshold} sumamente bajo para indicar una potencial presencia
        de la enfermedad. La información proveída por el modelo podría servir como un tipo de \textit{screening} temprano, seguido
        de un régimen establecido de exámenes para continuar con el monitoreo del paciente. En caso que la tecnología se vea limitada
        las unicas opciones fueran \textit{tratamiento} o \textit{no tratamiento}, seguiría considerando peor un \textbf{Falso Negativo}
        aunque se deberían de examinar los efectos secundarios o demás consecuencias del tratamiento.

      \item \textit{Basado en lo anterior, si tuviera que optimizar el modelo priorizando una sola métrica entre
        \textbf{Precisión (Precision)} y \textbf{Sensibilidad (Recall / Sensitivity)}, ¿cuál escogería y por qué?}

      \item \textit{¿Por qué el Accuracy sería una métrica peligrosa para presentar a los inversionistas en este
        caso específico?}
    \end{enumerate}

\end{document}
